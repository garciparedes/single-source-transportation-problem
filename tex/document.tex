% !TEX root = ./document.tex

\documentclass[a4paper, spanish]{article}

\usepackage{mystyle}
\usepackage{myvars}
\usepackage{mylinearprogramming}
%-----------------------------


\begin{document}

  \maketitle

  \begin{abstract}
    \noindent [TODO]
  \end{abstract}

  %-----------------------------
  %  TEXT
  %-----------------------------

  \section{Introducción}
  \label{sec:introduction}

    \paragraph{}
    El \emph{Problema de Transporte} se considera bien conocido y existe una extensa literatura dedicada a él dado que surge de manera natural en un gran número de fenómenos de distinta índole. El problema básico se basa en la búsqueda de la mejor planificación de abastecimiento sobre un conjunto de destinos a partir de un conjunto de fuentes de origenes, asumiendo un determinado coste unitario así como unas restricciones de capacidad y demanda para cada punto de origen y destino respectivamente.

    \paragraph{}
    Matemáticamente este problema se puede modelar como un problema de programación lineal de la siguiente manera: Asumiremos que existen $n$ puntos de destino cada uno de ellos con una demanda $d_j, \ \forall j \in \{1,...,n\}$, que deben ser abastecidas a partir de $m$ puntos de origen, con una capacidad denotada por $s_{i}, \forall i \in \{1, ..., m\}$. Además, existe un determinado coste asociado $c_{i, j}$ por llevar una unidad de producto entre el origen $i$ y el destino $j$. Para resolver este problema se utiliza una matriz de variables de decisión positivas de dimensión $n \cdot m$ donde la variable $x_{ij}$ determina la cantidad de producto enviado del origen $i$ al destino $j$. El modelo matemático para este problema se define en la ecuación \eqref{eq:transportation-model}.

    \begin{eqfloat}
      \begin{equation}
        \begin{array}{ll@{}ll}
          \text{Minimizar}	& \displaystyle \sum\limits_{i=1}^{m}\sum\limits_{j=1}^{n}c_{ij} \cdot x_{ij} \\
          \text{sujeto a}		& \sum\limits_{j=1}^{n} x_{ij}	\leq s_{i}, & \forall i \in \{1,...,m\}\\
                            &	\sum\limits_{i=1}^{m} x_{ij}	\geq d_{j}, & \forall j \in \{1,...,n\} \\
                            & x_{ij}	\geq 0, 	& \forall i \in \{1,...,m\}, \forall j \in \{1,...,n\}
        \end{array}
      \end{equation}
      \caption{Formulación básica del \emph{Problema de Transporte}.}
      \label{eq:transportation-model}
    \end{eqfloat}

    \paragraph{}
    El modelo de transporte de la ecuación \eqref{eq:transportation-model} tiene una importante propiedad que cumplen algunos problemas de programación lineal denominada \emph{Propiedad de Integridad}, que asegura que si todas las constantes del problema (costes, capacidades y demandas máximas) son enteras, entonces las variables de decisión tomarán valores enteros en en la solución óptima.

    \paragraph{}
    Nótese que este problema se puede aplicar a un gran número de situaciones que se dan en la vida real. Desde el caso más destacado, consistente en planificar el abastecimiento de mercancias entre puntos de origen y destino para minimizar costes, a otras como la diversificación de proyectos entre distintos equipos de trabajo en una empresa para reducir costes.

    \paragraph{}
    Sin embargo, el problema básico descrito en la ecuación \eqref{eq:transportation-model} es muy restringido a la hora de modelar situaciones reales más complejas. Por tanto, en este caso se va a describir una ampliación de este que permite limitar el número de fuentes de origen para cada destino (en este caso a una única fuente).

    \paragraph{}
    La descripción de esta nueva restricción se llevará a cabo en la Sección \ref{sec:approach}. Después, en la Sección \ref{sec:results} se aplicará el problema a un caso concreto con $m = 8$ orígenes y $n = 12$ destinos. Finalmente, se comentarán distintas cuestiones acerca de esta restricción en la Sección \ref{sec:conclusions}. Por último, en el Apéndice \ref{appendix:source-code} se incluye el código fuente utilizado para la realización de las distintas pruebas, mientras que en el Apéndice \ref{appendix:data} se incluyen los datos concretos utilizados para verificar el correcto funcionamiento del modelo descrito.

  \section{Planteamiento: Restricción de Fuente Única}
  \label{sec:approach}

    \paragraph{}
    La restricción de \emph{fuente única}, tal y como se ha indicado previamente, consiste en reducir la cantidad de orígenes que van a abastecer a cada destino en un único origen. Esta restricción tiene sentido y aplicaciones prácticas en el mundo real, ya que es común que en casos como este, se apliquen distintos costes administrativos al seleccionear más de una punto de origen para suministrar producto a cada destino. Esto puede hacer que determinadas soluciones que podrían parecer óptimas a priori, se convirtieran en inviables al aplicarlas al mundo real. En estos caso, puede ser más sencillo modelizar el problema de esta manera en lugar de tratar de recoger en el modelo todos los costes administrativos, lo cual complicaría demasiado el problema.

    \paragraph{}
    La adicción de esta nueva restricción puede ser vista por tanto como una nueva regla que convierta el problema en más difícil de resolver, por tener que comprobar un número mayor de vertientes así como de validez de restricciones en la búsqueda por la solución óptima. Por tanto, el modelo básico descrito en la ecuación \eqref{eq:transportation-model} es una relajación lineal respecto de este. De la misma manera, este problema tendrá soluciones contenidas en un subconjunto de soluciones del modelo básico. Es decir, si denotamos por $S_{T.}$ el conjunto de soluciones del problema básico y por $S_{T.F.U.}$ entonces se cumple que $S_{T.F.U.} \subseteq S_{T.}$. Nótese que la solución óptima obtenida con el modelo básico puede ser la misma que con el problema de \emph{fuente única}.

    \paragraph{}
    Para abordar esta nueva restricción existen dos modelizaciones principales, las cuales se describen a continuación, siendo la primera una extensión del modelo básico, lo cual añade más variables de decisión, y la segunda una simplificación del primero mediante.

    \subsection{Modelo 1}
    \label{sec:approach-1}

      \paragraph{}
      El \emph{Modelo 1} para la inclusión de la restricción de fuente única consiste en la ampliación del modelo básico mediante el uso de una matriz de decisión de variables binarias de dimensión $m \cdot n$ que denotaremos por $y_{ij}$ utilizadas para añadir posteriormente la restricción lógica de un único origen por destino. Para ello, es necesario que las variables binarias se \say{activen} cuando las variables $x_{ij}$ se activen, para ello utilizamos una cota superior basada en el mínimo entre el valor de origen y de destino, es decir, $min(s_{i}, d_{j})$. La restricción lógica de cantidad es trivial y consiste en limitar el número de orígenes por destino a un determinado valor (en este caso $1$ aunque podría ser otro diferente, o incluso un vector dependiente del destino). El modelo 1 se describe de manera completa en la ecuación \eqref{eq:single-source-transportation-model-1}.

      \begin{eqfloat}
        \begin{equation}
          \begin{array}{ll@{}ll}
            \text{Minimizar}	& \displaystyle \sum\limits_{i=1}^{m}\sum\limits_{j=1}^{n}c_{ij} \cdot x_{ij} \\
            \text{sujeto a}		& \sum\limits_{j=1}^{n} x_{ij}	\leq s_{i}, & \forall i \in \{1,...,n\}\\
                              &	\sum\limits_{i=1}^{m} x_{ij}	\geq d_{j}, & \forall j \in \{1,...,n\} \\
                              & x_{ij}	\leq min(s_{i}, d_{j}) \cdot y_{ij}, 	& \forall i \in \{1,...,m\}, \forall j \in \{1,...,n\} \\
                              &	\sum\limits_{i=1}^{m} y_{ij}	\leq 1, & \forall j \in \{1,...,n\} \\
                              & x_{ij}	\geq 0, 	& \forall i \in \{1,...,m\}, \forall j \in \{1,...,n\} \\
                              & y_{ij}	\in \{0, 1\}, 	& \forall i \in \{1,...,m\}, \forall j \in \{1,...,n\} \\
          \end{array}
        \end{equation}
        \caption{Formulación del \emph{Problema de Transporte de Fuente Única} siguiendo la \emph{Modelización 1}.}
        \label{eq:single-source-transportation-model-1}
      \end{eqfloat}

      \paragraph{}
      Tal y como se puede apreciar, al tratarse este modelo de una ampliación respecto del modelo básico, el número de restricciónes en este caso es mayor, incrementandose en $n + m \cdot n$. Además, son necesarias $m \cdot n$ nuevas variables de decisión de carácter binario, lo cual incrementa la dificultad del problema tal y como se indicó previamente.

    \subsection{Modelo 2}
    \label{sec:approach-2}

      \paragraph{}
      La modelización basada en el \emph{Modelo 2} consiste en una simplificación respecto del problema básico de transporte, dado que sustituye las variables de decisión positivas por variables de decisión binarias mediante la transformación $y_{ij} = x_{ij} / d_j$ lo cual se aplica a todas las restricciones del modelo y además, permite modificar la restricción de suministro mayor o igual, para añadir la igualdad a $1$, lo que permite que el problema sea de \emph{fuente única}. El modelo completo se define en la ecuación \eqref{eq:single-source-transportation-model-2}.

      \begin{eqfloat}
        \begin{equation}
          \begin{array}{ll@{}ll}
            \text{Minimizar}	& \displaystyle \sum\limits_{i=1}^{m}\sum\limits_{j=1}^{n}c_{ij} \cdot d_{j} \cdot y_{ij} \\
            \text{sujeto a}		& \sum\limits_{j=1}^{n} d_{j} \cdot y_{ij}	\leq s_{i}, & \forall i \in \{1,...,n\}\\
                              &	\sum\limits_{i=1}^{m} y_{ij}	= 1, & \forall j \in \{1,...,n\} \\
                              & y_{ij}	\in \{0, 1\}, 	& \forall i \in \{1,...,m\}, \forall j \in \{1,...,n\}
          \end{array}
        \end{equation}
        \caption{Formulación del \emph{Problema de Transporte de Fuente Única} siguiendo la \emph{Modelización 2}.}
        \label{eq:single-source-transportation-model-2}
      \end{eqfloat}

      \paragraph{}
      Tal y como se puede apreciar, la dificultad de este modelo es menor que la del Modelo 1 debido a que tiene menor número de variables de decisión. Sin embargo, este no es más sencillo que el problema básico sin restricciones de fuente única, ya que en este caso se están optimizando variables de carácter binario, lo cual implica una mayor difícultad para buscar soluciones factibles óptimas que en problemas de carácter continuo.

  \section{Resultados}
  \label{sec:results}

    \paragraph{}
    [TODO]

    \begin{table}
      \begin{center}
        \begin{tabular}{ | c | c | c | c | c | c | c | c | c | c | c | c | c | c |}
              \hline
          \multicolumn{2}{| c |}{\multirow{2}{*}{Coste Mínimo $c = 17003$}} &
            \multicolumn{12}{ c |}{Destinos $n = 12$} \\ \cline{3-14}
          \multicolumn{2}{| c |}{}
            & 1  & 2  & 3  & 4  & 5  & 6  & 7  & 8   & 9  & 10 & 11  & 12 \\ \hline
          \multirow{8}{*}{Orígenes $m = 8$}
            & 1 & 64 & 0  & 0  & 0  & 0  & 3  & 0  & 0   & 0  & 0  & 0   & 109 \\ \cline{2-14}
            & 2 & 0  & 0  & 0  & 88 & 0  & 0  & 29 & 0   & 0  & 0  & 0   & 0 \\ \cline{2-14}
            & 3 & 0  & 0  & 0  & 0  & 0  & 0  & 0  & 0   & 82 & 0  & 0   & 0 \\ \cline{2-14}
            & 4 & 0  & 0  & 0  & 0  & 0  & 0  & 39 & 0   & 0  & 0  & 113 & 0 \\ \cline{2-14}
            & 5 & 5  & 0  & 0  & 0  & 95 & 0  & 0  & 0   & 0  & 78 & 0   & 0 \\ \cline{2-14}
            & 6 & 0  & 0  & 95 & 1  & 0  & 9  & 0  & 0   & 0  & 0  & 0   & 0 \\ \cline{2-14}
            & 7 & 0  & 0  & 0  & 0  & 0  & 15 & 0  & 112 & 0  & 0  & 0   & 0 \\ \cline{2-14}
            & 8 & 0  & 98 & 0  & 0  & 0  & 77 & 0  & 0   & 0  & 0  & 0   & 0 \\ \hline
        \end{tabular}
      \end{center}
      \caption{Solución óptima para el problema aplicando la relajación lineal de varias fuentes.}
      \label{table:multi-source-optimal-solution}
    \end{table}

    \begin{table}
      \begin{center}
        \begin{tabular}{ | c | c | c | c | c | c | c | c | c | c | c | c | c | c |}
              \hline
          \multicolumn{2}{| c |}{\multirow{2}{*}{Coste Mínimo $c = 21942$}} &
            \multicolumn{12}{ c |}{Destinos $n = 12$} \\ \cline{3-14}
          \multicolumn{2}{| c |}{}
            & 1  & 2  & 3  & 4  & 5  & 6  & 7  & 8   & 9  & 10 & 11  & 12 \\ \hline
          \multirow{8}{*}{Orígenes $m = 8$}
            & 1 & 69 & 0  & 0  & 0  & 0  & 104 & 0  & 0   & 0  & 0  & 0   & 0 \\ \cline{2-14}
            & 2 & 0  & 0  & 0  & 89 & 0  & 0   & 68 & 0   & 0  & 0  & 0   & 0 \\ \cline{2-14}
            & 3 & 0  & 0  & 0  & 0  & 0  & 0   & 0  & 0   & 82 & 0  & 0   & 109 \\ \cline{2-14}
            & 4 & 0  & 0  & 0  & 0  & 0  & 0   & 0  & 0   & 0  & 0  & 113 & 0 \\ \cline{2-14}
            & 5 & 0  & 0  & 0  & 0  & 95 & 0   & 0  & 0   & 0  & 78 & 0   & 0 \\ \cline{2-14}
            & 6 & 0  & 0  & 95 & 0  & 0  & 0   & 0  & 0   & 0  & 0  & 0   & 0 \\ \cline{2-14}
            & 7 & 0  & 0  & 0  & 0  & 0  & 0   & 0  & 112 & 0  & 0  & 0   & 0 \\ \cline{2-14}
            & 8 & 0  & 98 & 0  & 0  & 0  & 0   & 0  & 0   & 0  & 0  & 0   & 0 \\ \hline
        \end{tabular}
      \end{center}
      \caption{Solución óptima para el problema aplicando la restricción de fuente única.}
      \label{table:single-source-optimal-solution}
    \end{table}

  \section{Conclusiones}
  \label{sec:conclusions}

    \paragraph{}
    [TODO]


  \begin{appendices}

    \section{Código Fuente}
    \label{appendix:source-code}

      \subsection{Problema de Transporte con Fuente Única: Relajación Lineal}
      \label{appendix:source-code-relaxation}

        \inputminted{text}{./../mosel/single-source-transportation-problem-relaxation.mos}

      \subsection{Problema de Transporte con Fuente Única: Modelo 1}
      \label{appendix:source-code-model-1}

        \inputminted{text}{./../mosel/single-source-transportation-problem-model-1.mos}

      \subsection{Problema de Transporte con Fuente Única: Modelo 2}
      \label{appendix:source-code-model-2}

        \inputminted{text}{./../mosel/single-source-transportation-problem-model-2.mos}

    \section{Datos}
    \label{appendix:data}

      \inputminted{text}{./../mosel/data.dat}

  \end{appendices}


  %-----------------------------
  %  Bibliographic references
  %-----------------------------

  \nocite{subject:pent2017}
  \nocite{tool:xpress-mosel}
  \nocite{repository:network-flow-transeuro}

  \bibliographystyle{alpha}
  \bibliography{bib}

\end{document}
